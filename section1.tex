\section{Introduction}

The interesting semilinear elliptic differential equation
\begin{equation}\label{eq:1.1}
  \Delta u + f(u) = 0, \quad x\in \mathbb{R}^n,
\end{equation}
arises in many areas of applied mathematics including astrophysics,
fluid mechanics, and population genetics.
If a solution exists in the whole space $\mathbb{R}^n$, satisfying
\begin{equation}\label{eq:1.2}
  u(x) \to 0 \quad \text{as} \quad |x| \to \infty,
\end{equation}
it is called a ground state.
One natural question to ask is whether the ground state is unique or not.
This is an extremely difficult problem to tackle in general.
The classical work of \textsc{Gidas}, \textsc{Ni}, \& \textsc{Nirenberg}
[4, 5] tells us that with some mild conditions on $f$, all ground states are
radially symmetric. This allows us to shift our study to the ordinary differential equation
\begin{equation}\label{eq:1.3}
  u'' + \frac{n-1}{r} u' + f(u) = 0,\quad r>0
\end{equation}

\begin{equation}\label{eq:1.4}
  u'(0) = 0,\quad u(r)\to 0\quad \text{as} \quad r\to\infty.
\end{equation}

Still this is a sufficiently difficult problem that few general results are known.
One exception is the recent result of \textsc{Peletier} \& \textsc{Serrin} [13],
which provides uniqueness for those $f$ that satisfy a starlike condition for large $u$
and are essentially more negative than positive for small $u$.
See also [6] for an improvement on this result.

\textsc{Coffman} [3] established uniqueness for the ground state of the equation
with particular choice of $n=3$ and
\begin{equation}\label{eq:1.5}
  f = u^3 - u.
\end{equation}
The result for the problem on the whole of $\mathbb{R}^n$ is deduced from that
of the corresponding problem on a finite interval $[0,b]$.
The main part of the proof is a study of the zeros of the function
\begin{equation}\label{eq:1.6}
  w(x) = \frac{\partial u(x)}{\partial \alpha},
\end{equation}
where $u$ is considered as a function of both $r$ and the parameter $\alpha=u(0)$.
Very clever identities are used to show that $w(x)$ changes sign exactly once in $[0,b]$.
The required conclusion then follows.
\textsc{Coffman} attributed the approach, especially the use of special identities,
to \textsc{Kolodner} [8]. \textsc{Kolodner} was concerned with a more general 
nonlinear eigenvalue problem (rather than just the uniqueness of the ground state)
for some sublinear eigenvalue equation arising from the study of the rotation of a heavy string.
\textsc{Coffman} remarked that the proof he gave in [3] did not extend to other
choices of $n$ or other powers of $u$ in $f$ since some of the clever identities
no longer hold.

When $f$ takes the more general from
\begin{equation}\label{eq:1.7}
  f = u^p - u,\quad p>1,
\end{equation}
there is the added complication of the non-existence of solutions if $p$ exceeds
or is equal to the critical value $(n+2)/(n-2)$; see, for instance, [1, 2].

Improving on \textsc{Coffman}'s method, \textsc{Mcleod} \& \textsc{Serrin} [10]
were able to establish a rather general result that includes \textsc{Coffman}'s.
In particular, for the special $f$ above, uniqueness holds for
\begin{alignat*}{2}
  & p<\infty & \quad & (1\leq n\leq 2), \\
  & p \leq \frac{n}{n-2} && (2<n\leq 4), \\
  & p<\frac{8}{n} && (4<n<8).
\end{alignat*}
Slightly sharper reaults are also available for $p$ close to $1$.

In this paper the expected result for all values of $p$ up to the critical exponent
is confirmed. In addition, we also study the ground state solutions for a ball,
an annulus, and the exterior of a ball.
In the cases when the origin is not included, uniqueness is established for all $p>0$.

The approach used in this paper is basically the same as that of \textsc{Coffman}.
Instead of using \textsc{Kolodner}-type identities we resort to \textsc{Sturm} comparison
techniques in the oscillation theory of the linear second-order differential equations.
The two methods are essentially equivalent, but the latter seems to make the proofs
more transparent. In the present case, the main difficulty lies in the choice of the
suitable comparison equation and in the proof that the equation has the correct
oscillation behavior. See the survey paper [9] for a more detailed explanation
of our modification of \textsc{Coffman}'s method and its use to obtain
earlier uniqueness results.

For some recent success of \textsc{Sturm}'s comparison technique in the study
of another property of the Emden-Flower equation, see [7].

In the spirit of \textsc{Coffman}'s paper, the result is presented only for the
particular choice of $f$ given by \eqref{eq:1.7}.
We refrain from stating the theorem for more general $f$
(although it appears not hard to do so), in order to keep the ideas of the proof clear.