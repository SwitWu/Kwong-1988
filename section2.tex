\section{Some Sturmian Theory}

The form of Sturm's comparison theorem found in most textbooks is not strong
enough for our purpose. The following formulation can be found, for instance,
on p. 229 in Ince's classical book, \emph{Ordinary Differential Equations}, Dover, 1956.
By a solution we always mean a nontrivial solution.

\begin{lemma}[Sturm]\label{lemma:1}
  $U$ and $V$ are, respectively, solutions of the following equations:
  \begin{equation}\label{eq:2.1}
    U''(x) + f(x)U'(x) + g(x)U(x) = 0,\quad x\in (a,b),
  \end{equation}
  \begin{equation}\label{eq:2.2}
    V''(x) + f(x)V'(x) + G(x)V(x) = 0,\quad x\in (a,b),
  \end{equation}
  where $f$, $g$, and $G$ are continuous (actually, only local integrability is needed).
  Let $(\mu,\nu)$ be a subinterval in which $V(x)\neq 0$ and $U(x)\neq 0$,
  and in which the comparison condition
  \begin{equation}\label{eq:2.3}
    G(x) \geq g(x)\quad \text{for all } x\in (\mu,\nu) 
  \end{equation}
  holds. Suppose further that
  \begin{equation}\label{eq:2.4}
    \frac{V'(\mu)}{V(\mu)} \leq \frac{U'(\mu)}{U(\mu)}.
  \end{equation}
  Then
  \begin{equation}\label{eq:2.5}
    \frac{V'(x)}{V(x)} \leq \frac{U'(x)}{U(x)}\quad \text{for all } x\in (\mu,\nu).
  \end{equation}
  Equality in \eqref{eq:2.5} can occur only if $U\equiv V$ in $[\mu, x]$.
  If either $\mu$ or $\nu$ is a zero of $U$ or $V$, then the functions
  in \eqref{eq:2.4} and \eqref{eq:2.5} are interpreted as $\infty$.
\end{lemma}

From this lemma follows the usual assertion that between any two consecutive
zero of $U(x)$, there must be at least one zero of $V(x)$.
In fact, $V(x)$ may have more than one zeros in that interval.
It is natural to say that $V(x)$ oscillates faster than $U(x)$.

Sometimes it is more convenient to replace the comparison conditions \eqref{eq:2.1},
\eqref{eq:2.2} and \eqref{eq:2.3} by the equivalent conditions
\begin{equation}\label{eq:2.1prime}
  U''(x) + f(x)U'(x) + h(x)U(x) \geq 0, \quad U(x)>0 \quad \text{in } (\mu,\nu), \tag{2.1$'$}
\end{equation}
\begin{equation}\label{eq:2.2prime}
  V''(x) + f(x)V'(x) + h(x)V(x) \leq 0, \quad V(x)>0 \quad \text{in } (\mu,\nu), \tag{2.2$'$}
\end{equation}
If either $U$ or $V$ is negative, the corresponding inequality has to be reversed.

A simple consequence of Sturm's theorem is the following.

\begin{lemma}\label{lemma:2}
  (i) If in addition to all the hypotheses of Lemma~\ref{lemma:1}, we have
  \begin{equation}
    0 < V(\mu) < U(\mu),
  \end{equation}
  then
  \begin{equation}
    V(x) \leq U(x)\qquad \text{for all } x\in (\mu,\nu).
  \end{equation}
  (ii) If in addition to all the hypotheses of Lemma \ref{lemma:2},
  $\lim_{x\to \nu} V(x) = \pm\infty$, then
  $\lim_{x\to\nu} U(x) = \pm\infty$.
\end{lemma}

\begin{proof}
  The conclusion follows from an integration of (2.5).
\end{proof}

Suppose that equation (2.1) has solutions that do not vanish
in a neighborhood of the point $b$.
The largest neighborhood of $b$, $(c,b)$, on which there exists
a solution of (2.1) without zeros is called the disconjugacy
neighborhood of $b$, or, in short, the disconjugacy interval
of (2.1). It follows from Sturm's theorem that no non-trivial
solution can have more than one zero in $(c,b)$.
On the other hand, unless $c=a$, any solution of (2.1)
that has a zero before $c$ must have another zero in $(c,b)$.
The following is another corollary of Sturm's theorem.

\begin{lemma}\label{lemma:3}
  Consider the same two equations, (2.1) and (2.2), satisfying
  the comparison condition (2.3).
  In addition, we assume that $U\not\equiv V$
  in any neighborhood of $b$. If there exists a solution $V$
  of (2.2) with a largest zero at point $\varrho$,
  then the disconjugacy interval of (2.1) is a strict superset of $(\varrho, b)$.
\end{lemma}

We need another comparison lemma that handles equations with different second terms.
It is not usually included in the classical Sturm theory.
It can be proved easily using results from the theory of differential inequalities.
See [9] for the proof of a related result.

\begin{lemma}\label{lemma:4}
  Suppose instead of (2.2), $V$ sartisfies the differential equation
  \begin{equation}\label{eq:2.8}
    V''(x) + F(x)V'(x) + G(x)V(x) = 0,\quad x\in (a,b),
  \end{equation}
  and the comparison condition
  \begin{equation}\label{eq:2.9}
    F(x) \geq f(x) \geq 0\quad \text{for all } x\in (\mu,\nu).
  \end{equation}
  The conclusions of Lemma~\ref{lemma:1} or Lemma~\ref{lemma:2}
  still hold provided that either
  \begin{equation}\label{eq:2.10}
    U'(x)\geq 0 \quad \text{for all } x\in (\mu,\nu)
  \end{equation}
  or
  \begin{equation}\label{eq:2.11}
    V'(x)\geq 0 \quad \text{for all } x\in (\mu,\nu).
  \end{equation}
\end{lemma}

\begin{proof}
  Let us assume that \eqref{eq:2.10} holds;
  the proof for the case in which \eqref{eq:2.11} holds is similar.
  Define $r(x) = U'(x)/U(x)\geq 0$ and $R(x) = V'(x)/V(x)$.
  They satisfy
  \[r'(x) = -(f(x)r(x)+g(x)+r^2(x)) \geq -(F(x)r(x)+G(x)+r^2(x)),\]
  and
  \[R'(x) = -(F(x)R(x)+G(x)+R^2(x)).\]
  We also have the initial comparison condition $r(\mu) \geq R(\mu)$.
  From the theory of differential inequalities, we can then conclude
  that $r(x)\geq R(x)$ for all $x$ in $(\mu,\nu)$.
\end{proof}

The next lemma, though not a direct consequence of Sturm's theorem,
is definitely motivated by it. It is well known that if $U$ is a solution
of a second-order ordinary differential equation such as \eqref{eq:2.1},
then at a zero of $U$, the derivative of $U'$ cannot vanish.
If $z$ satisfies a second-order ordinary differential inequality in such
a way that $z$ oscillates more than $U$,
then intuition tells us that the tangent line to the graph of $z$
is being bent more strongly towardz the $x$-axis than that of $U$.
Hence the derivative of $z$ at a zero cannot vanish.
A convenient reference for a proof of this fact has not been found.
Reproduced here is the short proof that was given in [7].

\begin{lemma}\label{lemma:5}
  Suppose a function $z(t)$ is positive (negative) in an interval
  $(\mu,\nu)$, either $z(\mu)=0$ or $z(\nu)=0$, and it satisfies
  the inequality
  \begin{equation}\label{eq:2.12}
    z''(t) + f(t)z'(t) + g(t)z(t) \leq (\geq ) 0,\quad \text{in }(\mu,\nu),
  \end{equation}
  where $f$ and $g$ are any continuous functions.
  Then $z'(\mu)\neq 0$ ($z'(\nu)\neq 0$).
\end{lemma}

\begin{proof}
  We give the proof only for the case $z(t)\geq 0$ and $z(\mu) = 0$;
  the other cases can be proved similarly. We may assume without loss of
  generality that $f(x)=0$, since an equation of the more general form
  can be reduced to this particular case using a change of independent variable.
  Equation \eqref{eq:2.12} now takes the form
  \begin{equation}\label{eq:2.13}
    z'' + g(x)z = -P(x),\quad x>0,
  \end{equation}
  where $P(x) \geq 0$. Let $z_1(x)$ and $z_2(x)$ be independent solutions of the
  homogeneous equation associated with different equation \eqref{eq:2.13},
  satisfying the initial conditions
  \begin{align*}
    & z_1(\mu) = 0,\qquad z_1'(\mu) = 1, \\
    & z_2(\mu) = 1,\qquad z_2'(\mu) = 0.
  \end{align*}
  Suppose that the conclusion of the lemma is false, namely,
  that $z'(\mu)=0$. By the formula of variation of constants
  \begin{equation}\label{eq:2.14}
    z(x) = \int_\mu^x [z_1(s)z_2(x) - z_1(x)z_2(s)] P(s)\,ds
  \end{equation}
  Let $h(s,x) = [z_1(s)z_2(x) - z_1(x)z_2(s)]$. At $x=s$,
  $\partial h(s,x)/\partial x$, being the Wronskian of $z_1(x)$
  and $z_2(x)$, is $-1$. By continuity $\partial h(s,x)/\partial x$
  is therefore negative in a neighborhood of $(0,0)$.
  At $x=s$, $h(s,x)=0$. Thus for $x>s$ and $x$ sufficiently close to $s$,
  $h(s)<0$. It then follows that the integrand in \eqref{eq:2.14}
  is negative for $x>\mu$ but sufficiently close to $\mu$.
  This then implies that $z(x)$ is negative, contradicting our assumption
  that it is positive in $(\mu,\nu)$.
\end{proof}

If the inequality sign in \eqref{eq:2.12} is strict at the point $x=\mu$,
then the conclusion follows trivially from the facts that $z$ has a minimum
at $\mu$ so that $z''(\mu)\geq 0$, contradicting \eqref{eq:2.12}.
The proof for Lemma \ref{lemma:5} is needed to handle the general case.

Next are established some asymtotic properties of the solutions of the linear
equation
\begin{equation}\label{eq:2.15}
  U'' + \frac{m}{x}U' + g(x)U = 0,\quad x>0,
\end{equation}
where $m>0$ is a constant and $g$ is continuous with $\lim_{x\to\infty} g(x) = -2k^2$,
for some $k>0$.

\begin{lemma}\label{lemma:6}
  Let $(c, \infty)$ be the disconjugacy interval of $(2.15)$. Every solution of (2.15) with a zero in $(c, \infty)$ is unbounded.
  
  Conversely, if the last zero of an unbounded solution of (2.15) is $\varrho$, then $\varrho$ is an interior point of the disconjugacy interval. In the other words, $\varrho>c$.
\end{lemma}

\begin{proof}
  We have to show that equation (2.15) is non-oscillatory, so that the disconjugacy interval is well defined. Let $\mu$ be so large that $g(x) \leq-k^2$, and $m / x \leq 2 k$ for $x \in[\mu, \infty)$. We compare (2.15) with the equation
  \[
  V^{\prime \prime}+2 k V^{\prime}-k^2 V=0, \quad x \in[\mu, \infty)
  \]
  using Lemma \ref{lemma:4} in the form of Lemma \ref{lemma:2}. We can conclude that for the solutions $U_1$ and $V_1$ of (2.15) and (2.16), respectively, that satisfy the initial conditions
  \[
  U_1(\mu)=V_1(\mu)=1, \quad U_1^{\prime}(\mu)=V_1^{\prime}(\mu)=1,
  \]
  the inequality
  \[
  V_1(x) \leq U_1(x), \quad x>\mu,
  \]
  holds. Since $V_1$ is non-oscillatory and unbounded, so is $U_1$. If all solutions of (2.15) are unbounded, then our lemma is true. Thus suppose there is a bounded solution $U_2$. Then $U_1$ and $U_2$ are linearly independent, and all other solutions are of the form $U=c_1 U_1+c_2 U_2$, with constants $c_1$ and $c_2$. It follows that any solution that is not a multiple of $U_2$ is unbounded. Let us show that $U_2$ cannot have a zero in the disconjugacy interval, from which the assertion of the lemma follows. Suppose that $U_2$ does have a zero at $\varrho \in(c, \infty)$. Let $U$ be the solution of (2.15) that satisfies the initial conditions, $U(\varrho+1)=U_2(\varrho+1)$ and $U^{\prime}(\varrho)=U_2^{\prime}(\varrho)-\varepsilon$ with $\varepsilon>0$. By taking $\varepsilon$ small enough, we ensure that $U$ has a zero so close to $\varrho$ that it falls in the disconjugacy interval. Thus $U$ cannot have another zero beyond $\varrho+1$. In other words, $U$ remains positive in $[\varrho+1, \infty)$. By Lemma $2, U(x) \leq$ $U_2(x)$, for $x \geq \varrho+1$. It follows that $U$ is bounded, but this contradicts the fact that $U$ is not a multiple of $U_2$.
\end{proof}

\begin{lemma}\label{lemma:7}
  If $U$ is a solution of $(2.15)$ such that $U^{\prime}(x)<0$ for all $x$ sufficiently large, then $-U^{\prime}(x) / U(x) \geq k$ for all $x$ sufficiently large.
\end{lemma}

\begin{proof}
  Let $R(x)=-U^{\prime}(x) / U(x)$. It satisfies the Riccati equation
  \[
  R^{\prime}=R^2-\frac{m}{x} R+g(x)<R^2+g(x) .
  \]
  If $R(x)<k$ for some large $x$, for which $g(x)$ is close to its limit $-2 k^2$, then $R^{\prime}$ will remain strictly negative, eventually causing $R$ to change sign. This contradiction proves the lemma.
\end{proof}