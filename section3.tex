\section{Classification of Solutions}\label{sec:3}

In this section and the next, we study our differential equation in an interval 
not containing the origin, in order to avoid having a singular term in the equation.
Let $p>1$ and $m \geq 0$ be any constants, and $(a, b)$ be a bounded or unbounded 
open subinterval of $[0, \infty)$, with $a>0$. We are concerned with the following
boundary value problems:

\begin{align}
  u^{\prime \prime}(r)+\frac{m}{r} u^{\prime}(r)+u^p-u & =0, \quad u(r)>0,
    \quad r \in(a, b), \label{eq:3.1} \\
  u^{\prime}(a) & = 0, \label{eq:3.2} \\
  u(b) & = 0
\end{align}

or

\begin{equation}\label{eq:3.4}
  \lim _{r \rightarrow \infty} u(x)=0 \quad \text { if } b=\infty .
\end{equation}

These come from corresponding boundary value problems for the semilinear equation 
(1.1) on an annulus $a<|x|<b$, with the Neumann condition on the ball $|x|=a$ and 
the Dirichlet condition on the outer ball $|x|=b$. The constant $m$ is one less 
than the dimension $n$ of the Euclidean space $\mathbb{R}^n$ in which (1.1) holds.
In this section, we take $m$ to be any positive constant, not necessarily integral.
Our main theorem asserts the uniqueness of the solution to any of the above boundary
value problems. Since the question of existence has been answered in the affirmative,
a unique solution always exists.

In this section we discuss the classification of solutions of (3.1). 
Some of the lemmas stated here are well known, but we prefer to give complete proofs.
The proof of the main theorem is complicated enough that it is perhaps better to make
the paper as self-contained as possible. Trying to adopt notations from other sources
would only add to the confusion.

Following \textsc{Kolodner} and \textsc{Coffman}, instead of considering directly the
boundary value problems, we look at $u$ as the solution of an initial value problem, 
(3.1), together with the initial conditions
\begin{equation}\label{eq:3.5}
  u(a)=\alpha>0, \quad u^{\prime}(a)=0.
\end{equation}
The solution $u$ now depends on $\alpha$ as a parameter.
In addition, $u$ depends on the constant $m$. Besides, 
we think of $u=u(r, \alpha, m)$ as defined for all values of $r>0$ 
extending beyond the interval $[a, b]$.

In other words, we are shooting out a solution $u(r, \alpha, m)$ from the point $r=a$
and hope that with the correct choice of the initial height $\alpha$,
the solution will land on the right spot $r=b$.
Rather than regarding $b$ to be a preassigned fixed point,
we let $b(\alpha, m)$ be the first point where $u(r, \alpha, m)$ intersects the
$r$-axis, and study the dependence of $b$ on $\alpha$ and $m$.
The point $b$ is not always defined. This happens, for instance, 
if for a fixed $m, \alpha$ is small enough; the solution $u(r, \alpha, m)$ will 
not cross the $r$-axis.

We divide the set of solutions into three subsets:
\begin{enumerate}
  \item Solutions that eventually takes on some negative values.
    These are the ones for which $b(\alpha, m)$ are defined.
    Let $N$ be the set of $(\alpha, m)$ for which the corresponding
    solutions belong to this class.
  \item Solutions that remain positive and satisfy
    $\lim _{r \rightarrow \infty} u(r)=0$. Let $G$ (stands for "ground state")
    be the set of $(\alpha, m)$ for which the corresponding solutions
    belong to this class.
  \item Solutions that remain non-negative but do not belong to case 2.
    Let $P$ (stands for "positive") be the corresponding set of $(\alpha, m)$.
    The term "positive" is justified because, as Lemma \ref{lemma:5} shows,
    no solutions of an ordinary differential equation can be tangent to the $r$-aixs.
\end{enumerate}

The sets $N, G$, and $P$ partition the "quadrant" $(0, \infty) \times[0, \infty)$
into three
mutually disjoint subsets. Sometimes, for convenience,
we say that a solution belongs to $N$,
$G$, or $P$ when we should have said that the corresponding $(\alpha, m)$ belongs to the
appropriate set. Also whenever there is no danger of confusion, we suppress the explicit
mention of the variables $\alpha$ and $m$ in functions such as $u$ and $b$.
From the classical theory of ordinary differential equations, we know that,
in any compact $r$-interval, solutions
of (3.1)--(3.5) depend continuously on the parameters $\alpha$ and $m$.

As is well known, we can obtain a lot of information about the solutions
from the Energy or Lyapunov function
\begin{equation}\label{eq:3.6}
  E(r)=\frac{u^{\prime 2}(r)}{2}+\frac{u^{p+1}(r)}{p+1}-\frac{u^2(r)}{2}  
\end{equation}

The inequality $E^{\prime}(r) = -m u^{\prime 2} / r \leq 0$ implies that $E$ decreases to a
finite constant $E(\infty)$ as $r \rightarrow \infty$. It follows that all solutions must be
bounded.

\begin{lemma}\label{lemma:8}
  If $E(a)=\frac{\alpha^{p+1}}{p+1}-\frac{\alpha^2}{2} \leq 0$, then $(\alpha, m) \in P \quad$ for any $m$. In other words, $(0, \gamma) \times[0, \infty) \subset P$, where $\gamma=\left(\frac{p+1}{2}\right)^{1 /(p-1)}$.
\end{lemma}

\begin{proof}
  Since $E$ is decreasing, $E(r) \leq E(a) \leq 0$, for $r>a$.
  The solution cannot cut the
  $r$-axis; otherwise at a point of intersection
  $E(r)=u^{\prime 2} / 2>0$ contradicting our
  previous assertion.
\end{proof}

\begin{lemma}\label{lemma:9}
  If $u \in P$, then it is oscillatory about the value 1 in the following sense.
  The sets of its local maxima $\left\{a=r_0<r_2<\ldots\right\}$
  and local minima $\left\{r_1<\right.$ $\left.r_3<\ldots\right\}$ interlace and, 
  unless $u \equiv 1$
  \begin{equation}\label{eq:3.7}
    u\left(r_0\right)>u\left(r_2\right)>u\left(r_4\right)>\ldots>1    
  \end{equation}
  and
  \begin{equation}\label{eq:3.8}
    u\left(r_1\right)<u\left(r_3\right)<u\left(r_5\right)<\ldots<1.    
  \end{equation}
\end{lemma}


\begin{proof}
  We claim that $u$ cannot be monotone. Suppose this is not true.
  We know that $u$ has a local maximum at $r=a$, because $u^{\prime \prime}(a)<0$.
  Thus $u$ can only be monotone decreasing.
  Let the limit $\lim _{r \rightarrow \infty} u(x)$ be denoted by $u(\infty)$.
  We have
  \[
    \lim _{r \rightarrow \infty} u^{\prime \prime}(r)
      = \lim _{r \rightarrow \infty}\left(-\frac{m}{r} u^{\prime}-u^p+u\right)
      = -u^p(\infty)+u(\infty).
  \]
  This is incompatible with the monotonicity of $u$ unless the limit is zero,
  which implies that $u(\infty)=1$. Let $v=u-1>0$. It satisfies the differential equation.
  \[
    v^{\prime \prime}+\frac{m}{r} v^{\prime}+\left(\frac{u^p-u}{u-1}\right) v = 0.
  \]
  Note that the fraction inside the parentheses is
  larger than or equal to $p-1$ (for $u>1$).
  We use Sturm's comparison theorem to conclude that $v$ oscillates
  faster than the solutions of
  \[
    U^{\prime \prime}+\frac{m}{r} U^{\prime}+(p-1) U=0 .
  \]
  It is well known that all solutions $U$ of this differential equation are oscillatory,
  but this contradicts the fact that the more oscillatory $v$ remains positive
  for all $r$.

  Hence we know that $u$ has at least one local minimum. Let $r_1$ be the first of them.
  We cannot have $u\left(r_1\right)=1$, for then $E\left(r_1\right)$ attains the lowest
  possiblie value that $E$ can take. This means that $E(r)=E\left(r_1\right)$ for all $r>r_1$,
  and we have the case $u \equiv 1$ In all other cases, since $r_1$ is a local minimum,
  $u^{\prime}\left(r_1\right)=0$ and 
  $u^{\prime \prime}\left(r_1\right)=u\left(r_1\right)-u^p\left(r_1\right) \geq 0$.
  This can hold only if $u\left(r_1\right)<1$. It follows that $E\left(r_1\right)<0$.
  The same arguments as above show that $u$ cannot be monotone increasing
  in $\left(r_1, \infty\right)$.
  Repeating the arguments then gives two infinite sequences of critical points as asserted.
  The strict inequalities in (3.7) and (3.8) are due to the fact
  that $E^{\prime}=-u^{\prime 2} / 2$ cannot vanish identically
  in any subinterval of $(a, b)$.
\end{proof}

Lemma \ref{lemma:9} has the following corollary


\begin{lemma}\label{lemma:10}
  Suppose a solution does not belong to $N$. Then it belongs to $G$ if and only if $E(\infty)=0$, and it belongs to $P$ if and only if $E(\infty)<0$.
\end{lemma}

\begin{proof}
  If $u \in G$, then $\lim _{r \rightarrow \infty} u(r)=\lim _{r \rightarrow \infty} u^{\prime}(r)=0$. Thus $E(\infty)=0$. On the other hand, if $u \in P$, then $E\left(r_1\right)<0$ because $u\left(r_1\right)<1$. Thus $E(\infty)<E\left(r_1\right)$ $<0$.
\end{proof}

\begin{lemma}\label{lemma:11}
  Let $u$ belong to either $N$ or $G$. Then $u^{\prime}(r)<0 \quad$ in $(a b(\alpha, m)]$ or $(a, \infty)$, respectively.
\end{lemma}

\begin{proof}
  Suppose $u^{\prime}(c)$ vanishes in an interior point $c \in(a, b)$. Then $u(c) \neq 1$, lest $u \equiv 1$. If $u(c)<1$, then $E(c)<0$, contradicting the assumption that $u$ belongs to $N \cup G$. Finally, if $u(c)>1$, then $u^{\prime \prime}(c)<0$, so that $c$ is a local maximum. Between the two local maxima $r=a$ and $c$ there must be a local minimum, at which $u^{\prime \prime} \geq 0$, and this reduces to the previous case.
\end{proof}

The derivative $u^{\prime}(b)$ cannot vanish at the endpoint $b$,
because this contradicts the uniqueness of initial value problem for (3.1);
the trivial solution is the only solution of (3.1) that can vanish at $b$
with a double zero.


\begin{lemma}\label{lemma:12}
  For $(\alpha, m) \in N$, $b(\alpha, m)$ is a continuous function of $\alpha$ and $m$.
\end{lemma}

\begin{proof}
  This is a simple consequence of the fact that a solution $u$ can never be tangent to the $r$-axis at a zero.
\end{proof}

\begin{lemma}\label{lemma:13}
  The sets $N$ and $P$ are open subsets of $(0, \infty) \times[0, \infty)$.
\end{lemma}

\begin{proof}
  Let $\left(\alpha_0, m_0\right)$ be a point in $N$. The corresponding solution $u$ takes a negative value at some point $r=c$. By continuity, there is a neighborhood of $\left(a_0, m_0\right)$ for which $u(c, \alpha, m)<0$ for each $(\alpha, m)$ in the neighborhood. Thus all such solutions belong to $N$. Let $\left(a_0, m_0\right)$ be a point in $P$. Then $E(c)<0$ for some $r=c$. Continuity again yields a neighborhood for which $E(c)<0$ for each of the solutions belonging to the neighborhood. It follows that $E(\infty)<E(c)<0$ for each of these solutions, and so they must all belong to $P$.
\end{proof}

An immediate consequence of this lemma is that for each fixed $m$,
the sets $N_m$ and $P_m$ (the intersections of $N$ and $P$ with 
the straight line $(., m)$) are open subsets of $(0, \infty)$. 
Each is therefore a union of countably many open intervals.
The boundary points of $N$ or $P$ belong to $G$.
The following lemma is a straightforward consequence of Lemmas 8 and 13.

\begin{lemma}\label{lemma:14}
  For a fixed m, the boundary value problem (3.1)-(3.4) has a unique solution if and only if
  the sets $P_m$ and $N_m$ are both open intervals of the form $\left(0, \alpha_m\right)$ and
  $\left(\alpha_m, \infty\right)$, respectively,
  with one single common endpoint $\alpha_m \in G_m$.
\end{lemma}

We call a point $m$ regular if it satisfies the conditions of Lemma \ref{lemma:14}.
For each regular $m$, there is therefore a unique $\alpha(m)=\alpha_m$ as asserted.
This is, in fact, a continuous function of $m$ within any interval of regularity.
As a first step towards establishing that fact, we prove that give any fixed $\bar{m}>0$,
the set of points in $G$ with $m \leq \bar{m}$ is bounded.

\begin{lemma}\label{lemma:15}
  For any given $\bar{m}>0$, there exists an $\bar{\alpha}>0$
  such that $G\subset (0, \bar{\alpha})\times[0, \bar{m}]$.
\end{lemma}

\begin{proof}
  It suffices to show that for some $\bar{\alpha}$,
  $[\bar{\alpha}, \infty) \times[0, \bar{m}] \subset N$. Consider the comparison equation
  \begin{equation}\label{eq:3.9}
  U^{\prime \prime}+\frac{\bar{m}}{a} U^{\prime}+U^p-U=0 \quad \text { on }[a, \infty) .
  \end{equation}
  Since the coefficient of the second term is a constant, the equation can be explicitly
  solved, at least in theory. It is then not hard to see that there exists an $\bar{\alpha}>0$, 
  such that if $U(a) \geq \bar{\alpha}, U^{\prime}(a)=0$, then $U(r)$ cuts the $r$-axis at 
  some point, say $b$.
  Using as new variables $S=U^{\prime 2}(r)$ and $x=U(r)$, we can rewrite \eqref{eq:3.9} as
  \begin{equation}\label{eq:3.10}
    \frac{d S}{d x}=\frac{2 \bar{m}}{a} \sqrt{S}+f(x), \quad S(\alpha)=0,
  \end{equation}
  where $f(x)=2\left(x^p-x\right)$. Similarly, we can rewrite $(3.1)$,
  using $s=u^{\prime 2}(r)$ and $x=u(r)$, as
  \begin{equation}\label{eq:3.11}
    \frac{d s}{d x}=\frac{2 m}{r} \sqrt{s}+f(x)
      \leq \frac{2 \bar{m}}{a} \sqrt{s}+f(x), \quad s(\alpha)=0.
  \end{equation}

  An application of the theory of differential inequalities gives
  \begin{equation}\label{eq:3.12}
    S(x) \leq s(x), \quad \text { for all } 0 \leq x \leq \alpha.    
  \end{equation}
  In other words, $U^{\prime 2}\left(r_1\right) \leq u^{\prime 2}\left(r_2\right)$, 
  whenever $x=U\left(r_1\right)=u\left(r_2\right)$; see Figure 1.
  Since both $u^{\prime}$ and $U^{\prime}$ are negative, we have
  \begin{equation}\label{eq:3.13}
    0>U^{\prime}\left(r_1\right) \geq u^{\prime}\left(r_2\right).    
  \end{equation}
  This implies that $u$ must cut the $r$-axis before $b$. It follows that $(\alpha, m) \in N$.
\end{proof}

\begin{lemma}\label{lemma:16}
  If all points $m \in\left(m_1, m_2\right)$ are regular, then $\alpha_m$ is a continuous
  function of $m$. The set of limit points of the curve $\left\{\alpha(m): m_1<m<m_2\right\}$
  that lie on each of the lines $m=m_1$ and $m=m_2$ are thus connected closed intervals.
  In case one of the endpoints is also regular, the corresponding limit set reduces to the
  single point $\alpha\left(m_1\right)$ or $\alpha\left(m_2\right)$.
\end{lemma}

\begin{proof}
  Let $m_0 \in\left(m_1, m_2\right)$. Suppose $\lim _{m \rightarrow m_0} \alpha(m)$ is not $\alpha\left(m_0\right)$. By compactness, there exists a sequence of points $\left\{m_{(i)}\right\} \subset\left(m_1, m_2\right), \quad m_{(i)} \rightarrow m_0$, but $\alpha_{\infty}=$ $\lim _{i \rightarrow \infty} \alpha\left(m_{(i)}\right) \neq \alpha\left(m_0\right)$. Then $\alpha_{\infty}$ belongs either to $N$ or to $P$, contradicting the fact that both are open sets. The other assertions are obvious.
\end{proof}