\section{Main Result, \boldmath{$a>0$}}\label{sec:4}

The main result is established through a careful study of the signs of the function
values in $[a,b(\alpha,m))$ of the function
\begin{equation}\label{eq:4.1}
  w(r,\alpha,m) = \frac{\partial u}{\partial \alpha} (r,\alpha,m).
\end{equation}
We are interested only in those $w$ that come from solutions in either $N$ or $G$,
and we are concerned only with their behavior in the interval $[a, b$ )
(we interpret $b$ to be $\infty$ in case $u \in G$ ).
When there is no danger of confusion, we suppress the variables $\alpha$ and $m$ in the
function notation.

By differentiating \eqref{eq:3.1} and \eqref{eq:3.5},
we derive the following initial value problem for $w$:
\begin{equation}\label{eq:4.2}
  w^{\prime \prime}+\frac{m}{r} w^{\prime}+\left(p u^{p-1}-1\right) w=0,
\end{equation}
\begin{equation}\label{eq:4.3}
  w(a)=1, \quad w^{\prime}(a)=0.
\end{equation}
Equation \eqref{eq:4.2} is a ``linear'' equation in $w$ if we regard $u$ as a known function.
An immediate consequence is that $w$ cannot be tangent to the $r$-axis.
As $r$ goes through a zero of $w, w$ has to change sign.

\begin{lemma}\label{lemma:17}
  For $u \in G \cup N, w$ has to change sign at least once in $[a, b)$.
\end{lemma}

\begin{proof}
  Rewrite \eqref{eq:3.1} as
  \begin{equation}\label{eq:4.4}
    (u-1)^{\prime \prime}+\frac{m}{r}(u-1)^{\prime}+\left(\frac{u^p-u}{u-1}\right)(u-1) = 0,
  \end{equation}
  and view it as a ``linear'' equation in $(u-1)$, with the expression inside the large
  parentheses as the coefficient of the last term. Since this coefficient is smaller than the
  corresponding one in (4.2), when $u \geq 1$, we conclude from Sturm's comparison theorem
  that $w$ oscillates faster than $(u-1)$, so $w$ must vanish before the first zero of $(u-1)$,
  namely, before the point $\xi \in(a, b)$ at which $u(\xi)=1$.
\end{proof}

Our ultimate aim is to show that $w$ cannot change sign more than once in $[a, b)$.
We say that $(\alpha, m)$ is admissible if this is true, more precisely,
when
\begin{equation}\label{eq:4.5}
  (\alpha, m) \in G \cup N, \quad \text{and}
    \quad w(r, \alpha, m) \text{ has exactly one zero in } [a, b).
\end{equation} 
If, in addition,
\begin{equation}\label{eq:4.6}
  w(b, \alpha, m)<0, \quad \text { for }(\alpha, m) \in N,
\end{equation}
or
\begin{equation}\label{eq:4.7}
  \lim _{r \rightarrow \infty} w(r, \alpha, m)=-\infty, \quad \text { for }(\alpha, m) \in G,  
\end{equation}
we say that $(\alpha, m)$ is strictly admissible. The following well known lemma connects the
concepts of admissibility and regularity and is the key idea in the
\textsc{Kolodner}-\textsc{Coffman} method.

\begin{lemma}\label{lemma:18}
  Let $m$ be fixed. If $(\alpha, m) \in N$ is strictly admissible,
  then in a neighborhood of $\alpha, b$ is a strictly decreasing function of $\alpha$.
\end{lemma}

\begin{proof}
  By assumption, $\frac{\partial u}{\partial \alpha}(b(\alpha), \alpha)=w(b(\alpha))<0$.
  Hence, for $\varepsilon>0$ small enough, $u(b(\alpha), \hat{\alpha})<0$
  for all $\hat{\alpha} \in(\alpha, \alpha+\varepsilon)$. By the Intermediate Value theorem,
  $u(r, \hat{\alpha})$ must have a zero in $(a, b(\alpha))$.
  Therefore, $b(\hat{\alpha})<b(\alpha)$.
  It is not hard to see that this local monotonicity together with the continuity
  of $b$ (Lemma \ref{lemma:12}) implies monotonicity in a neighborhood of $\alpha$.
\end{proof}

The following lemma complements Lemma \ref{lemma:18}.

\begin{lemma}\label{lemma:19}
  If for $(\alpha, m) \in G, \lim _{r \rightarrow \infty} w(r)=-\infty(\infty)$,
  in particular if $(\alpha, m)$ is strictly admissible,
  then there exists a right (left) neighborhood of a that belongs to $N$.
\end{lemma}

\begin{proof}
  Let $\tau$ be the largest zero of $w$ in $[a, b)$.
  By Lemma \ref{lemma:6}, the disconjugacy interval $(c, \infty)$ of equation~\eqref{eq:4.2} contains $\tau$. 
  Suppose first that $w$ is negative after $\tau$. Take a point $\zeta \in(c, \tau)$ and a 
  point $\lambda \in(\tau, \infty)$. Since $w(\zeta)>0$ and $w(\lambda)<0$, for $\varepsilon>0$ 
  small enough, $u(\zeta, \hat{\alpha})>w(\alpha)$, and $u(\lambda, \hat{\alpha})<w(\alpha)$, 
  for all $\hat{\alpha} \in(\alpha, \alpha+\varepsilon)$. In case $w$ is positive beyond 
  $\tau$, the same is true for all $\hat{\alpha}$ in a left neighborhood
  $(a-\varepsilon, \alpha)$ instead. So $\hat{u}(r)=u(r, \hat{\alpha})$ must intersect $w(r)$ 
  at a point $\chi$ in $(\zeta, \lambda)$. This point is therefore within the disconjugacy 
  interval $(c, \infty)$. We claim that $\hat{u}(r)$ must intersect the $r$-axis
  in $(\chi, \infty)$. Suppose that this is not true. Then $u(r, \hat{\alpha})$
  remains positive in $(\chi, \infty)$. It lies below the graph of $w$ in a right neighborhood 
  of the point $\chi$. Let us suppose that it catches up with $w$ at a point $\eta>\chi$.
  Then in $(\chi, \eta)$, the function $z(r)=u(r)-\hat{u}(r)$ is positive and satisfies the 
  differential equation
  \begin{equation}\label{eq:4.8}
    z^{\prime \prime}+\frac{m}{r} z^{\prime}+\left(\frac{u^p-\hat{u}^p}{u-\hat{u}}-1\right) z=0.    
  \end{equation}
  Observe that the coefficient of the last term is less than that in \eqref{eq:4.2}.
  Hence $z$ oscillates less than the solutions of \eqref{eq:4.2}.
  But the disconjugacy interval of \eqref{eq:4.2}
  being $(c, \infty)$ means that the solutions of $(4.2)$ cannot have two zeros in the interval 
  $[\chi, \eta]$. It follows that the less oscillatory $z$ cannot have two zeros in the same 
  interval, contradicting the assumption that $z(\chi)=z(\eta)=0$. Thus $\hat{u}$ remains below 
  $u$ throughout $(\chi, \infty)$.
  Then the coefficient of \eqref{eq:4.8} is less than that of \eqref{eq:4.2}
  in the whole interval $(\chi, \infty)$, implying that $z$ is less oscillatory than the 
  solutions of \eqref{eq:4.2} in $(\chi, \infty)$. Since $\chi$ is in the disconjugacy interval
  of \eqref{eq:4.2}, by Lemma \ref{lemma:6},
  a solution of \eqref{eq:4.2} that vanishes at $\chi$ must be unbounded.
  By Lemma \ref{lemma:2}, the less oscillatory $z$ must also be unbounded.
  This contradicts the fact that 
  $\hat{u}(r)<u(r) \rightarrow 0$ as $r \rightarrow \infty$, unless $\hat{u}$ intersects the 
  $r$-axis. Thus $u(r)=u(r, \bar{\alpha})$ belongs to $N$ for all $\bar{\alpha}$
  in $(\alpha, \alpha+\varepsilon)$ or $(\alpha-\varepsilon, \alpha)$ as asserted.
\end{proof}

Now if we can show that all points $(\alpha, m)$ in the union $G \cup N$ are strictly 
admissible, then we have uniqueness for the boundary value problems
\eqref{eq:3.1}, \eqref{eq:3.2}, \eqref{eq:3.3}
and \eqref{eq:3.1}, \eqref{eq:3.2}, \eqref{eq:3.4}.
Let us follow the development as $\alpha$ increases from 0.
By Lemma~\ref{lemma:8}, we first have solutions in $P$.
As the first boundary point $\alpha_m$ of $G$ is reached,
we have a solution in $G$. By Lemma \ref{lemma:19}, a right neighborhood of $\alpha_m$
belongs to $N$. By Lemma \ref{lemma:18}, as $\alpha$ continues to increase from $\alpha_m$,
the point $b(\alpha)$ moves strictly towards the lefthand side without retracing
any previous locations. Thus all $\alpha>\alpha_m$ belongs to $N$, and no two values of
$\alpha$ can solve the boundary value problem
\eqref{eq:3.1}, \eqref{eq:3.2}, \eqref{eq:3.3} for the same $b$.

The desired claim (that all members of $G \cup N$ are strictly admissible), although valid,
is by no means easy to verify.

We introduce the function
\begin{equation}\label{eq:4.9}
  \theta(r)=-r u^{\prime}(r) / u(r), \quad r \in[a, b),
\end{equation}
for a solution $u \in G \cup N$.

\begin{lemma}\label{lemma:20}
  The function $\theta$ is continuous in $[a, b), \theta(a)=0$
  and $\lim _{r \rightarrow b} \theta(r)=\infty$.
\end{lemma}

\begin{proof}
  The last assertion for the case $b=\infty$ is the only non-trivial one.
  By Lemma 7, $-r u^{\prime}(r) / u(r)>k r$ for large $r$. The conclusion then follows.
\end{proof}

We draw a horizontal line of height $\beta \geq 0$ above the $r$-axis. It can interact with the graph of $\theta$ in any one of the ways depicted in Figures 2--4.

With the given $\beta$, we define the function
\begin{equation}\label{eq:4.10}
  v_\beta(r)=r u^{\prime}(r)+\beta u(r), \quad r \in[a, b].  
\end{equation}
The next lemma relates the properties of $v$ to the way $\theta$ interacts
with the horizontal line. We omit the simple proof.
The last assertion is nothing more than the elementary Inverse Function theorem.

\begin{lemma}\label{lemma:21}
  $v(r)=(>,<) 0$ if and only if $\theta$ intersects (is below, is above)
  the straight line at $r$.

  $v$ is tangent to the $r$-axis at $r$ if and only if $\theta$ is tangent to the line at $r$.
  
  The smallest zero of $v$ (the first intersection point of $\theta$ with the straight line), 
  denoted by $\varrho=\varrho_\beta$ is a non-decreasing function of $\beta$.
  If $\theta^{\prime}\left(\varrho_\beta\right) \neq 0$, then the function $\varrho$ is 
  continuous at $\beta$.
\end{lemma}

It is easy to verify that $v$ satisfies the differential equation
\begin{equation}\label{eq:4.11}
  v^{\prime \prime}+\frac{m}{r} v^{\prime}+\left(p^{p-1}-1\right) v 
    = -2\left(u^p-u\right)+\beta(p-1) u^p.
\end{equation}
Let us define
\begin{equation}\label{eq:4.12}
  \phi(u)=-2\left(u^p-u\right)+\beta(p-1) u^p .  
\end{equation}
The right-hand side of \eqref{eq:4.11} is then the following composite function
\[\Phi(r)=\phi(u(r)).\]

\begin{lemma}\label{lemma:22}
  For any $\beta \geq 0$, there exists a unique point $\sigma=\sigma_\beta \in[a, b)$
  such that
  \begin{equation}\label{eq:4.13}
    \Phi(r)<0 \text { for } r<\sigma \text { and } \Phi(r)>0 \text { for } r>\sigma.  
  \end{equation}
  The point $\sigma$ is a continuous non-decreasing function of $\beta$.
\end{lemma}

\begin{proof}
  If $\beta(p-1)-2 \geq 0$, then $\sigma=a$. If $\beta(p-1)-2<0, \phi(u)$
  is a concave function of $u$. It therefore has exactly one zero $u_0$ distinct from 0.
  Then $\sigma$ is the unique (since $u$ is strictly decreasing) point 
  for which $u(\sigma)=u_0$ in case $u_0<\alpha$ and is $a$ otherwise.
  The continuity of $\sigma$ follows from the continuity of $u_0$ on $\beta$
  and the continuity of the inverse of $u$. Finally, the monotonicity of $\sigma$
  follows from the positivity of the term containing $\beta$ in \eqref{eq:4.12}.
\end{proof}

Let us follow the movement of $\varrho$ and $\sigma$ as $\beta$ increases.
For $\beta=0, \varrho=0$ and $\sigma>0$. Thus at this starting moment,
the point $\varrho$ is positioned to the left of the point $\sigma$.
As we increase $\beta, \varrho$ moves to the right while $\sigma$ moves to the left.
For $\beta$ large enough, $\varrho$ will be very close to $b$ while $\sigma$
will be very close to $a$. Thus a switching of the positions of $\varrho$ and $\sigma$
is bound to occur. If it is not for the possibility of a discontinuous jump of $\rho$,
we can conclude at once that for some suitable value of $\beta$,
the two points $\varrho$ and $\sigma$ meet head-on. Nevertheless,
this assertion is still true.

\begin{lemma}\label{lemma:23}
  There exists a $\beta_0>0$ at which $\varrho=\sigma$.
\end{lemma}

\begin{proof}
  We need to affirm the impossibility of a discontinuity of $\varrho$ (by Lemma \ref{lemma:21},
  we have only to show that $\theta^{\prime}(\varrho) \neq 0$ ) before the meeting
  of the points takes place. Suppose, then, $\beta>0$ is such that
  $\varrho_\beta \leq \sigma_\beta$.
  By the definition of $\sigma, \Phi(r) \leq 0$ for $r \leq \sigma_\beta$.
  In particular, $\Phi(r) \leq 0$ in the interval $\left[a, \varrho_\beta\right]$.
  Hence, in this interval, $v$ is non-negative and satisfies the differential inequality
  \begin{equation}\label{eq:4.14}
    v^{\prime \prime}+\frac{m}{r} v^{\prime}+\left(p u^{p-1}-1\right) v \leq 0.
  \end{equation}
  By Lemma \ref{lemma:5}, $v^{\prime}\left(\varrho_\beta\right) \neq 0$. This is equivalent,
  by Lemma \ref{lemma:21}, so the fact $\theta^{\prime}\left(\varrho_\beta\right) \neq 0$, as desired.
\end{proof}

What do $\theta$ and $v$ and their properties have to do with our goal?
We answer this question with the following lemma.

\begin{lemma}\label{lemma:24}
  If $\theta$ intersects the straight line $\theta=\beta_0$ exactly once,
  then the corresponding $u$, or more precisely, the corresponding $(\alpha, m)$
  is strictly admissible.
\end{lemma}

\begin{proof}
  The hypotheses imply that $v=v_{\beta_0}$ has exactly one zero in $(a, b)$.
  Note that in case $b \neq \infty, v(b)=b_0 u^{\prime}(b)<0$.
  Thus $v$ has exactly one zero in $[a, b]$. In $(a, \varrho)$,
  $v$ is positive and satisfies inequality \eqref{eq:4.14},
  while $w$ satisfies the equation \eqref{eq:4.2}.
  At $a$,
  \[
    v^{\prime}(a) = \left(1+\beta_0\right) u^{\prime}(a)+\beta_0 a u^{\prime \prime}(a)
      = \beta_0 a u^{\prime \prime}(a) < 0,
  \]
  so that the comparison conditions for the initial point is verified.
  By Lemma~\ref{lemma:1} , $v$ oscillates faster than $w$, so that $\varrho$,
  the first zero of $v$ is less than $\tau$, the first zero of $w$.
  In the interval $(\tau, b)$, $v$ is negative and satisfies the reverse differential inequality
  \begin{equation}\label{eq:4.15}
    v^{\prime \prime}+\frac{m}{r} v^{\prime}+\left(p u^{p-1}-1\right) v \geq 0.
  \end{equation}
  Thus $v$ again oscillates faster than $w$ in $(\tau, b)$. By assumption $v$
  has no more zeros beyond $\tau$, so $w$ cannot have any zero beyond $\tau$ either.
  In case $b<\infty$, we conclude that $w(b) \neq 0$, and so $(\alpha, m)$
  is strictly admissible. When $b=\infty$, using Lemma \ref{lemma:3},
  we see that the disconjugacy interval for \eqref{eq:4.2} is a proper superset of $(\tau, b)$.
  By Lemma~\ref{lemma:6}, $w$ is thus unbounded, and $(\alpha, m)$ is strictly admissible.
\end{proof}

Our next lemma tells us that we can even start out with less and yet end up
with more than we expect.

\begin{lemma}\label{lemma:25}
  Let $\xi$ be the only point in $(a, b)$ at which $u(\xi)=1$. If $\theta(r) \geq \beta_0$
  for all $r \in[\varrho, \xi]$, then $\theta^{\prime}(r)>0$ for all $r \in[a, b)$.
  As a result the corresponding $(\alpha, m)$ is strictly admissible.
\end{lemma}

\begin{proof}
  We first prove that $\theta^{\prime}(r) \geq 0$ in $[\xi, b)$ in any case.
  Suppose this is not true. Then there exist local minima in $(\xi, b)$,
  since $\lim _{r \rightarrow b} \theta(r)=\infty$.
  We draw a horizontal line of height $\beta$ to touch the lowest of all such minima,
  say at the point $c \in(\xi, b)$.
  The corresponding $v_\beta$ is then negative in $(c, b)$ and has a double zero at $r=c$.
  In $[\xi, b), \Phi(r) \geq 0$, so $v_\beta$ satisfies differential inequality \eqref{eq:4.15}.
  This contradicts Lemma \ref{lemma:5}.
  
  Together with the hypotheses, we now have $\theta(r) \geq \beta_0$
  for all $r \in[\varrho, b)$.
  As already established in the proof of Lemma \ref{lemma:23},
  $\theta^{\prime}(r)>0$ for $r \in[a, \varrho]$.
  Let us derive a contradiction by assuming that $\theta^{\prime}(r)<0$ for $r>\varrho$.
  Under this assumption, $\theta$ has local minima in $(\varrho, b)$.
  Since $\theta(r) \geq \beta_0$ in $(r, b)$, such minima have height above $\beta_0$.
  Let us raise the horizontal line from height $\beta_0$ to a height $\beta$
  to touch the smallest of these local minima, say at a point $c>Q$.
  Then $v_\beta$ is negative in $(c, b)$, and has a double zero at $c$.
  Since $\beta \geq \beta_0, \varrho_\beta \geq \sigma_\beta$,
  implying that $\Phi(c) \geq 0$. Again this contradicts Lemma \ref{lemma:5}.
\end{proof}

For convenience, we say that the point $(\alpha, m)$ is normal
if the hypotheses of Lemma \ref{lemma:25} is satisfied. In particular, if $\theta$ is monotone,
or alternatively if $\theta^{\prime} \geq 0$, then $(\alpha, m)$ is normal.
Such a $\theta$ must in fact be strictly montone. An important corollary of Lemma \ref{lemma:25}
is that normality is ``contagious''.

\begin{lemma}\label{lemma:26}
  Suppose that $(\bar{\alpha}, \bar{m}) \in G \cup N$ is normal.
  There exists a neighborhood of $(\bar{\alpha}, \bar{m})$ within which all members of $G \cup N$ are normal.
\end{lemma}

On the other hand, any limit point of a set of normal points is normal.

\begin{proof}
  Let $\bar{\xi}=\xi(\bar{\alpha})$. Pick a point $\eta \in(\bar{\xi}, b(\bar{\alpha}))$.
  We first find a neighborhood of $(\bar{\alpha}, \bar{m})$ small enough that
  for all its members $(\alpha, m), \xi(\alpha)<\eta<b(\alpha)$.
  This is possible because of the continuity of $\xi$ and $b$ on $\alpha$.
  By Lemma~\ref{lemma:25}, $\theta^{\prime}(r, \bar{\alpha})>0$ for all $r \in[a, \eta]$.
  The continuity of $\theta$ on $r, \alpha$, and $m$ implies that given any $r$ in this 
  interval, there exists a neighborhood of $r$ and a neighborhood
  of $(\bar{\alpha}, \bar{m})$ such that $\theta^{\prime}(r, \alpha, m)>0$
  for all $r$ and $(\alpha, m)$ in these neighborhoods.
  A compactness argument gives a neighborhood of $(\alpha, m)$
  in which $\theta^{\prime}>0$ for all $r \in[\alpha, \eta]$.
  By Lemma~\ref{lemma:25}, all members of $G \cup N$ in this neighborhood must be normal.

  For any sequence of normal points, $\theta^{\prime}>0$ for all $r \in[a, b)$.
  After taking limit, $\theta^{\prime} \geq 0$ for each point in the
  domain of the limit function. Thus the limit point must be normal.
\end{proof}

Ths lemma has some useful repercussions.
The first is the connectedness property of sets of normal points.

\begin{lemma}\label{lemma:27}
  Let $C$ be a connected subset of $G \cup N$. If one of its members is normal,
  then all it members are normal.
\end{lemma}

\begin{proof}
  The set of normal points in $C$ is both relatively open and relatively closed in $G \cup N$.
\end{proof}

\begin{lemma}\label{lemma:28}
  We fix an $m$. If $\bar{\alpha}$ is known to be normal,
  then there exists an $\alpha_0 \leq \bar{\alpha}$ such that
  $\alpha_0 \in G_m,\left(\alpha_0, \infty\right) \subset N_m$,
  and all $\alpha \geq \alpha_0$ are normal. For some $\varepsilon>0$,
  the interval $\left(\alpha_0-\varepsilon, \alpha_0\right)$ is disjoint from $G_m \cup N_m$.
\end{lemma}

\begin{proof}
  Let $C$ be the largest connected component of $G_m \cup N_m$ containing $\bar{\alpha}$.
  Since $C$ is a closed set, it has a least element $\alpha_0$.
  By Lemma \ref{lemma:27}, all members of $C$, in particular $\alpha_0$, are normal.
  By Lemma \ref{lemma:24}, each member of $C$, in particular $\alpha_0$, is strictly admissible,
  and so by Lemma \ref{lemma:19}, a right neighborhood of $\alpha_0$ belongs to $N$.
  By Lemma \ref{lemma:18}, the point $b(\alpha)$ is decreasing in $\alpha$.
  Thus as we increase $\alpha$ from $\alpha_0$, we remain in $N$.
  It follows that $\left(\alpha_0, \infty\right) \subset N$.
  The last assertion in the lemma is a consequence of the fact that
  a member of a connected component cannot be a limit point of other components.
\end{proof}

As required in the hypotheses of Lemma \ref{lemma:27}, we need to get hold of some normal points
before we can start things rolling. In fact, we have many normal points.

\begin{lemma}\label{lemma:29}
  For any $m \in[0,1]$, all members of $G \cup N$ are normal.
\end{lemma}

\begin{proof}
  In the interval $[a, \xi]$,
  \begin{equation}\label{eq:4.16}
    \left(-r u^{\prime}(r)\right)^{\prime} = -r u^{\prime \prime}(r)-u^{\prime}(r)
      = -(1-m) u^{\prime}(r)+r\left(u^p(r)-u(r)\right) \geq 0 .
  \end{equation}
  Thus $-r u^{\prime}(r)$ is non-decreasing in $r$ in $[a, \xi]$.
  Since $u$ is decreasing, the quotient $\theta(r)=-r u^{\prime}(r) / u(r)$ is
  non-decreasing in $[a,\xi]$. Thus the solution is normal.
\end{proof}

This lemma yields our main result for the values $m \in[0,1]$.
\textsc{McLeod} \& \textsc{Serrin} of course have obtained this part for the case 
$[a, b)=[0, \infty)$ using a different method. We are now ready to make some progress 
concerning other values of $m$.

\begin{lemma}\label{lemma:30}
  If all points $m \in[0, \infty)$ are regular, then our main result holds, namely,
  that all the boundary value problems we are interested in have unique solutions.
\end{lemma}

\begin{proof}
  By Lemma \ref{lemma:16}, if all $m$ are regular, the curve of the function $\alpha(m)$
  is continuous on $[0, \infty)$.
  The set $G \cup N$ coincides with the set $\{(\alpha, m): \alpha \geq \alpha(m)\}$
  of points on or above the curve. This is a large, connected piece.
  By Lemmas \ref{lemma:27} and \ref{lemma:29}, every member of this set is normal.
  By Lemma \ref{lemma:24}, they are all strictly admissible, and our main result follows.
\end{proof}

To establish our main result, we assume that the hypothesis of this lemma is not true and derive a contradiction. Thus let us suppose that there is a largest connected interval of regularity $[0, \bar{m})$, $\bar{m}<\infty$.

Let us first dispose of the possibility that $\bar{m}$ is regular. Suppose it is.
Then by Lemma \ref{lemma:16}, the curve of $\alpha(m)$ is continuous up to the endpoint $\bar{m}$.
There exists a sequence of irregular points $m_i>\bar{m}$ ($i=1,2,.$),
such that $\lim_{i \rightarrow \infty} m_i=\bar{m}$.
For each $i$, let $\alpha_l$ be the smallest member of $G$ on the line $m=m_i$.
Just as in the proof of Lemma \ref{lemma:16}, it can be shown that the sequence $\alpha_i$ 
cannot has a subsequence converging to a point in $G$ or $N$. 
It follows that $\lim _{i \rightarrow \infty} \alpha_i=\alpha(\bar{m})$. 
Some of these $\alpha_i$ must enter into the normality neighborhood 
of $(\alpha(\bar{m}),\bar{m})$ constructed in Lemma \ref{lemma:16}. 
Such $\alpha_i$ are therefore normal. 
By Lemma \ref{lemma:24}, all $\alpha>\alpha_i$ on the line $m=m_i$ must be in $N$. 
This contradicts the irregularity of $m_i$ with $\alpha_i$ being the smallest 
but not the unique member of $G$. It remains to show that the other possibility,
that $\bar{m}$ is irregular, is also void. It takes several more lemmas.
First of all, let us see how the sets $G_m, N_m$,
and $P_m$ look like on the line $L: m=\bar{m}$.
The curve of $\alpha(m)$ for $m<\bar{m}$ must be continuous up to the endpoint $\bar{m}$,
lest by Lemma \ref{lemma:16}, the limit point set of the curve on $L$ is a non-degenerate closed interval, 
a possibility excluded by Lemma \ref{lemma:18}. Let $\alpha_0=\lim _{m \rightarrow \bar{m}} \alpha(m)$.
By Lemma \ref{lemma:26}, $\alpha_0$ must be normal and, since it belongs to $G$,
must coincide with the $\alpha_0$ found in Lemma \ref{lemma:28}.
Hence the part of $L$ above $\alpha_0$ belongs to $N_{\bar{m}}$.
The set under the graph of $\alpha(m), m<\bar{m}$, shown as the shaded area in Figure 5,
belongs to $G$. The part of $L$ below $\alpha_0$, shown as the dotted line in the figure, 
cannot contain any members of the open set $N$, lest every neighborhood of such a member will 
intrude into the shaded area and therefore cannot be made up of points in $N$ alone.

\begin{lemma}\label{lemma:31}
  Let $\alpha_0$ be the largest member of $G_{\bar{m}}$ as described above, and $\alpha_1$ be the next member of $G$ smaller than $\alpha_0$. Let $\psi$ denote the largest zero of the function $w\left(r, \alpha_1\right)$. Then for all $\alpha<\alpha_0$ and $\alpha \neq \alpha_1, u(\psi, \alpha)<u\left(\psi, \alpha_1\right)$.
\end{lemma}

\begin{proof}
  By Lemma \ref{lemma:28}, $\alpha_0$ cannot be a limit point of smaller members of $G$. Thus $\alpha_1$ exists. The existence of a largest zero of $w$ (i.e., $w$ is non-oscillatory near $\infty$) follows from the fact that $w$ satisfies differential equation (4.2), which is of the form considered in Lemma \ref{lemma:6}. To simplify the notations, we use $u$ to represent the solution corresponding to the parameter $\alpha$ and $u_1$ that corresponding to $\alpha_1$. Suppose for some $\alpha$ the conclusion of the lemma is not true; in other words, we have instead
  \begin{equation}\label{eq:4.17}
    u(\psi) \geq u_1(\psi)
  \end{equation}

  We claim that the two solutions $u$ and $u_1$ cannot intersect at a point 
  in $(\psi, \infty)$. The arguments we use are similar to those in the proof of Lemma \ref{lemma:19}. 
  Suppose the solutions do intersect, for the first time at $r=\mu>\psi$.
  In view of (4.17), $u$ is below $u_1$ in a right neighborhood of the point $\mu$.
  Suppose the former catches up with the latter at some point beyond $\mu$.
  Let $\nu$ be the next point where the two solutions intersect again.
  In the interval $[\mu, v], u \leq u_1$ and the function $z=$ $u_1-u$ satisfies the 
  differential equation
  \begin{equation}\label{eq:4.18}
    z^{\prime \prime}+\frac{m}{r} z^{\prime}+\left(\frac{u_1^p-u^p}{u_1-u}-1\right) z=0.
  \end{equation}
  Notice that the coefficient in the last term is smaller than that in equation \eqref{eq:4.2}
  for $w\left(r, \alpha_1\right)$. Therefore, $w$ oscillates more than $z$.
  Since $z$ is zero at the endpoints of $[\mu, \nu]$, $w$ must have at least one zero in the 
  interval, contradicting the choice of $\psi$ as the last zero of $w$.
  Thus $u$ must remain below $u_1$ in $(\mu, \infty)$. In this interval,
  the coefficient of the last term of \eqref{eq:4.19} is smaller than that of \eqref{eq:4.2}.
  Hence the disconjugacy interval of \eqref{eq:4.19} is larger than that of \eqref{eq:4.2},
  which is $(\psi, \infty)$. Since $z$ has a zero within the disconjugacy interval,
  by Lemma~\ref{lemma:6}, $z$ is unbounded, contradicting the fact that $u$ is being trapped 
  between $u_1$ and the $r$-axis.

  Thus we have
  \begin{equation}\label{eq:4.19}
    u(r)>u_1(r), \text { for all } r>\psi    
  \end{equation}
  In the interval $[\psi, \infty)$, the same function $z$ as defined above satisfies
  \eqref{eq:4.18}, but this time the coefficient in the last term is larger than 
  that in \eqref{eq:4.2}. Since $z$ does not vanish in $[\psi, \infty)$, 
  the disconjugacy interval of \eqref{eq:4.18}, and hence also that of the less 
  oscillatory \eqref{eq:4.2}, is a proper superset of $[\psi, \infty)$.
  Hence, by Lemma \ref{lemma:6},
  \begin{equation}\label{eq:4.20}
    \lim _{r \rightarrow \infty} w\left(r, \alpha_1\right)=\pm \infty.  
  \end{equation}
  By Lemma \ref{lemma:19}, a right or left neighborhood of $\alpha$ belongs to $N$, contradicting the fact that no point below $\alpha_0$ can be in $N$.
\end{proof}

\begin{lemma}\label{lemma:32}
  The solutions $u_0=u\left(r, \alpha_0\right)$ and $u_1=u\left(r, \alpha_1\right)$ cannot 
  intersect more than once.
\end{lemma}

\begin{proof}
  The technique used in the proof of Lemmas \ref{lemma:19} and \ref{lemma:31} applies.
  The function $z=u_0-u_1$ satisfies the differential equation
  \begin{equation}\label{eq:4.21}
    z^{\prime \prime}+\frac{m}{r} z^{\prime}+\left(\frac{u_0^p-u_1^p}{u_0-u_1}-1\right) z=0,  
  \end{equation}
  which oscillates faster than \eqref{eq:3.1} for $u\left(r, \alpha_0\right)$
  but slower than \eqref{eq:4.2} for $w\left(r, \alpha_0\right)$, as long as $u_0>u_1$.
  Thus $z$ must have a zero in $[a, \infty)$, after the only zero $\tau$
  of $w\left(r, \alpha_0\right)$. Suppose $z$ has more than one zero,
  contrary to the conclusion of the Lemma. Then after the second zero,
  $\tau_2$, \eqref{eq:4.21} again oscillates more slowly than the equation for $w$.
  Since the disconjugacy interval of the equation for $w$ must be
  at least $[\tau, \infty)$, that of the "less oscillatory" \eqref{eq:4.21}
  must be at least $[\tau_2, \infty)$. By Lemma \ref{lemma:6}, $z$ is therefore unbounded,
  an obvious contradiction.
\end{proof}

\begin{lemma}\label{lemma:33}
  For all $\alpha \in\left(\alpha_1, \alpha_0\right)$, the solutions $u=u(r, \alpha)$ and
  $u_1=u\left(r, \alpha_1\right)$ cannot intersect more than once in $[a, \psi]$.
\end{lemma}

\begin{proof}
  As we vary the parameter $\alpha$ from $\alpha_0$ towards $\alpha_1$,
  we have a continuous deformation of the solution curve $u$
  over the closed interval $[a, \psi]$. At the left endpoint $r=a$,
  the curve $u$ stays clear above that of $u_1$, while at the right endpoint $r-\psi$,
  the former stays clear below the latter. We start out with one single point of intersection, 
  when $\alpha=\alpha_0$. The number can increase only if at some point $\alpha$,
  the curve of $u$ bulges up or down somewhere to touch the curve of $u_1$.
  But this is impossible because the function $z=u_1-u$ satisfies a ``linear'' second-order 
  differential equation, namel,y (4.19), and so cannot have a double zero.
\end{proof}

\begin{lemma}\label{lemma:34}
  The point $\left(\alpha_1, \bar{m}\right)$ is admissible.
\end{lemma}

\begin{proof}
  Since $\psi$ has been chosen to be the last zero of $w\left(r, \alpha_1\right)$,
  we need to show that $w$ has no other zeros before $\psi$, in order to
  satisfy the definition of admissibility.
  
  Let us first show that there cannot be more than one zero before $\psi$.
  Suppose this is not the case. Then for a point $\mu$ between the first and second zeros,
  $w(\mu)=$ $\partial u(u) / \partial \alpha<0$. Likewise for a point $\nu$
  between the second and third (which may be $\psi$ )
  zeros $w(\nu)=\partial u(\nu) / \partial \alpha>0$. We can choose an $\alpha>\alpha_1$ 
  sufficiently close to
  $\alpha_1$, that $u(\mu)>u_1(\mu)$ but $u(\nu)<u_1(\nu)$. By the Intermediate Value Theorem, 
  $u$ must intersect $u_1$ at least once in $(a, \mu)$ and another time in $(\nu, \phi)$, 
  contradicting Lemma \ref{lemma:33}.

  Next we show that $w$ cannot have exactly one zero before $\psi$. Suppose it does;
  then for $c$ between the two zeros of $w, w(c)<0$. 
  Hence, for $\alpha<\alpha_0$ but close to $\alpha_0, u(c)>u_1(c)$.
  It follows that $u$ must intersect $u_1$ at least twice,
  once in $(a, c)$ and once in $(c, \psi)$. A continuity argument as
  in the proof of Lemma \ref{lemma:33}
  shows that all solutions $u$ with $\alpha<\alpha_1$ must intersect $u_1$ at least twice. But 
  this contradicts the obvious fact that the solution $u(r, 1) \equiv 1$ intersects $u_1$ only 
  once.
\end{proof}


The last lemma we need turns out to be the most surprising.

\begin{lemma}\label{lemma:35}
  Admissibility implies normality.
\end{lemma}

\begin{proof}
  Suppose the point in question is not normal.
  Then the graph of $\theta$ intersects the line at height $\beta_0$ more than once
  in $[\varrho, \xi]$, as shown in Figure 6.
  Let $\zeta$ be the next point of intersection after $\varrho$.
  Using Lemma \ref{lemma:5} as in the proof of Lemma \ref{lemma:25},
  we see that $\theta$ cannot be tangent to the straight line at $\zeta$.
  We now lower the horizontal line to a height $\beta_1$ when it touches the first point 
  $\lambda$ to the right of $\zeta$ at which $\theta^{\prime}(\lambda)=0$.
  Our useful Lemma \ref{lemma:5} shows that $\sigma_{\beta_1}>\lambda$. In the interval $[\zeta,\lambda]$ 
  the function $\theta$ is strictly monotone, so that the inverse function is continuous.
  In other words, the point of intersection of the graph with the line $\theta=\beta$ is a 
  continuous function of $\beta$. As $\beta$ varies from $\beta_1$ to $\beta_0$ a switching of 
  positions of the point of intersection with the curve in $[\zeta, \lambda]$ and the point 
  $\sigma$ has taken place. Thus, by continuity, there is a $\beta_2$ at which $\sigma$ 
  coincides with the point of intersection in $[\zeta, \lambda]$. Let us focus on the line 
  $\theta=\beta_2$. In the interval $(\varrho, \sigma), \Phi(r)<0$ but $v$ is negative so that 
  equation \eqref{eq:4.11} oscillates less than equation \eqref{eq:4.2}.
  This imply that $w$ must have a zero in between the two zeros, $\varrho$ and $\sigma$, of $v$. 
  Since $\theta$ tend to $\infty$ near $b$, the graph of $\theta$ must intersect the line 
  one more time after $\sigma$, say at $x$. 
  In the interval $(\sigma, x), \Phi(r)>0$ but $v$ is positive so that again \eqref{eq:4.11}
  oscillates less than equation \eqref{eq:4.2}.
  As a result, $w$ must have another zero between the two zeros, 
  $\sigma$ and $x$ of $v$. This contradicts the admissibility of the solution.
\end{proof}


We can now complete the proof of our main theorem by observing that Lemmas \ref{lemma:34} and
\ref{lemma:35} imply that the point $\alpha_1$ is normal.
But by Lemma \ref{lemma:28}, the point $\alpha_0>\alpha_1$
will be in $N_{\bar{m}}$, obviously contradicting the definition of $\alpha_0$.